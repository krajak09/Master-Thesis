% https://portfoliooptimizationbook.com/

% \cite{palomar:portfolio_optimization}
\newpage
\subsection{Prices and Returns}

The price of an asset $p_t$ is the most obvious quantity, where $t$ is the discrete time index, it can also be continuous.

When it comes to modeling, using the logarithm of the prices enhances the mathematical convenience and also represents a wider dynamic range naturally.

% The simplest model for the log-prices is the random walk


\subsection{Financial Data: I.I.D. Modeling}
EMH: price of security = intrinsic value
any further information about future prospects incorporated in the current price (forecast = current price) 
-> modeling the prices as a random walk $\iff$ returns as a sequence of iid random variables

multiple assets: random variables = return of all assets (multivariate random variable)

under iid: no temporal structure incorporated returns are assumed to be independent distribution of random returns assumed fixed
\subsection{I.I.D. Model}
ignores temporal structures/ dependency in the data
\[x_t = \mu + \epsilon_t\]
\subsection{Sample Estimators}
sample mean
sample covariance matrix
estimators are unbiased and consistent
\subsection{Location Estimators}
\begin{enumerate}
    \item $\textbf{Least square estimator}$
    \item $\textbf{Median estimator}$
    \item $\textbf{Spatial median estimator or geometric median}$
\end{enumerate}

Numerical experiments

For $\textbf{Gaussian data}$, the sample mean is the best estimator (as further analyzed in Section 3.4), the spatial median is almost identical, and the elementwise median is the worst. 
For $\textbf{heavy-tailed data}$, the sample mean is as bad as the median and the spatial median remains the best. 

Overall, the spatial median seems to be the best option as it is robust to outliers and does not underperform under Gaussian data.

\subsection{Gaussian ML estimators}
Maximum likelihood estimation: important technique in estimation theory
likelihood function

pdf to the iid model, assuming that the residual follows a multivariate normal or Gaussian distribution, is 
\[f(x) = \frac{1}{\sqrt{(2\pi)^N|\Sigma|}}exp(-\frac{1}{2}(x-\mu)^\intercal\Sigma^{-1}(x-\mu))\]


\subsection{Autoregressive (AR) Model}
% 15. August 2025-------------------------------------------------------

% \textbf{Autoregressive (AR) model}

% \begin{itemize}
%     \item[\textbf{Order 1:}] $$x_t = \varphi_0 + \varphi_1x_{t-1}+\epsilon_t,$$ where $\varphi_0,\varphi_1$ are constants, $\epsilon_t$ we want to be normal distributed with mean 0 and variance $\sigma^2$ since we want to apply the Kalman filter. 
%     The state space model: 
    
%     state: $\begin{bmatrix} 1 \\ x_t
%     \end{bmatrix}$
    
%     transition matrix: $\begin{bmatrix}
%         1 & 0 \\ \varphi_0 & \varphi_1
%     \end{bmatrix}
%     \begin{bmatrix}
%         1 \\ x_t
%     \end{bmatrix} = 
%     \begin{bmatrix}
%         1 \\ \varphi_0 + \varphi_1
%     \end{bmatrix}$

%     covariance matrix: COV = $\begin{bmatrix}
%         0 (\sigma) & 0 \\ 0 & \sigma
%     \end{bmatrix}$
    
%     \item[\textbf{Order n:}] $$x_t= \varphi_0 + \sum_{i=1}^n\varphi_ix_{t-i} + \epsilon_t$$

%     state: $\begin{bmatrix}
%         1 \\ x_t \\ x_{t-1} \\ ... \\ x_{t-n}
%     \end{bmatrix}$

%     transition matrix: $\begin{bmatrix}
%         1 & 0 & ... & 0 & 0 \\
%         \varphi_0 & \varphi_1 & ... & \varphi_{n-1}& \varphi_n \\
%         0 & 1 & 0 & ... & 0 \\
%         0 & ... & ... & ... & 0 \\
%         0 & ... & ... & 0 & 1\\
%     \end{bmatrix}
%     \begin{bmatrix}
%         1 \\ x_t \\ x_{t-1} \\ ... \\ x_{t-n}
%     \end{bmatrix}=
%     \begin{bmatrix}
%         1 \\
%         \varphi_0 + \sum_{i=1}^n\varphi_ix_{t-i} \\
%         x_t\\ 
%         ...\\
%         x_{t-n+1}
%     \end{bmatrix}$
        
%     covariance matrix: $\begin{bmatrix}
%         0 & ... & ... & ... & 0 \\
%         0 & \sigma & 0 & ... & 0 \\
%         0 & ... & ... & ... & 0 \\
%         ... & ... & ... & ... &... \\
%         0 & ... & ... & ... & 0
%     \end{bmatrix}$
% \end{itemize}

%-----------------------------------------------------------------------
AR(1): $x_t = \varphi_0 + \varphi_1 x_{t-1} + \epsilon_t,$ where $\epsilon_t \sim \mathcal{N}(0, \sigma^2)$

\textbf{State vector}
\[
\mathbf x_t =
\begin{bmatrix}
1\\ x_t
\end{bmatrix}.
\]

\textbf{State transition}
\[
\mathbf x_t =
\underbrace{\begin{bmatrix}
1 & 0\\
\varphi_0 & \varphi_1
\end{bmatrix}}_{F}\,
\mathbf x_{t-1}
+
\underbrace{\begin{bmatrix}
0\\ \epsilon_t
\end{bmatrix}}_{\mathbf q_t},
\qquad
\epsilon_t \sim \mathcal N(0,\sigma^2).
\]

\textbf{Process covariance}
\[
Q \;=\;
\begin{bmatrix}
0 & 0\\
0 & \sigma^2
\end{bmatrix}.
\]

\textbf{Measurement equation}
\[
y_t = 
\underbrace{\begin{bmatrix} 0 & 1 \end{bmatrix}}_{H}\,\mathbf x_t + r_t,
\qquad
r_t \sim \mathcal N(0,R).
\]

\textbf{Initialization}
\[
\mathbf x_0 \sim \mathcal N(m_0, P_0).
\]

% AR(p) with intercept in state–space form
\noindent AR(p): $x_t = \varphi_0 + \sum_{i=1}^p \varphi_i x_{t-i} + \epsilon_t,$ where  $\epsilon_t \sim \mathcal{N}(0, \sigma^2)$

\textbf{State vector}
\[
\mathbf{x}_t =
\begin{bmatrix}
1 \\
x_t \\
x_{t-1} \\
\vdots \\
x_{t-p+1}
\end{bmatrix} \in \mathbb{R}^{p+1}.
\]

\textbf{transition matrix:}
\[
F =
\begin{bmatrix}
\varphi_0 & \varphi_1 & \varphi_2 & \cdots & \varphi_p \\
1 & 0 & 0 & \cdots & 0 \\
0 & 1 & 0 & \cdots & 0 \\
0 & 0 & 1 & \cdots & 0 \\
\vdots & \vdots & \vdots & \ddots & \vdots \\
0 & \cdots & 0 & 1 & 0 
\end{bmatrix}
\]

\textbf{observation matrix:}
\[
H =
\begin{bmatrix}
0 & 1 & 0 & \cdots & 0
\end{bmatrix}
\]

\textbf{process noise covariance:}
\[
Q =
\begin{bmatrix}
0 & 0 & \cdots & 0 \\
0 & \sigma^2 & \cdots & 0 \\
\vdots & \vdots & \ddots & \vdots \\
0 & 0 & \cdots & 0
\end{bmatrix}
\in \mathbb{R}^{(p+1)\times(p+1)}.
\]

\textbf{observation equation:}

\[
y_t =
\underbrace{\begin{bmatrix}
0 & 1 & 0 & \cdots & 0
\end{bmatrix}}_{\displaystyle H}
\mathbf{x}_t + r_t,
\qquad
r_t \sim \mathcal{N}(0,R).
\]

\textbf{State transition}
\[
\mathbf{x}_t =
\underbrace{
\begin{bmatrix}
\varphi_0 & \varphi_1 & \varphi_2 & \cdots & \varphi_p \\
1 & 0 & 0 & \cdots & 0 \\
0 & 1 & 0 & \cdots & 0 \\
0 & 0 & 1 & \cdots & 0 \\
\vdots & \vdots & \vdots & \ddots & \vdots \\
0 & 0 & \cdots & 1& 0 
\end{bmatrix}
}_{\displaystyle F}
\mathbf{x}_{t-1}
+
\underbrace{
\begin{bmatrix}
0 \\
\epsilon_t \\
0 \\
\vdots \\
0
\end{bmatrix}
}_{\displaystyle \mathbf{q}_t},
\qquad \epsilon_t \sim \mathcal{N}(0,\sigma^2).
\]



\textbf{Initialization}
Choose prior mean and covariance for the state
\[
\mathbf{x}_0 \sim \mathcal{N}(m_0,\; P_0).
\]


\subsection*{Markov Property of the AR(p) Model} % ----------------chatGPT----------------

The autoregressive model of order \(p\),
\[
x_t = \varphi_0 + \sum_{i=1}^{p} \varphi_i x_{t-i} + \epsilon_t,
\quad \epsilon_t \sim \mathcal{N}(0, \sigma^2),
\]
is not a first-order Markov process in its scalar form, since the value of \(x_t\) depends on the \(p\) previous values \(x_{t-1}, x_{t-2}, \dots, x_{t-p}\). Therefore, it is a Markov process of order \(p\).

However, when the AR(p) model is written in state-space form, we define a state vector
\[
\mathbf{x}_t =
\begin{bmatrix}
1 \\
x_t \\
x_{t-1} \\
\vdots \\
x_{t-p+1}
\end{bmatrix}
\in \mathbb{R}^{p+1},
\]
and the dynamics become:
\[
\mathbf{x}_t = A \mathbf{x}_{t-1} + \mathbf{w}_t,
\]
where \(\mathbf{w}_t\) is process noise.

In this formulation, the state process \(\{\mathbf{x}_t\}\) satisfies the first-order Markov property:
\[
P(\mathbf{x}_t \mid \mathbf{x}_{t-1}, \mathbf{x}_{t-2}, \dots) = P(\mathbf{x}_t \mid \mathbf{x}_{t-1}).
\]

\textbf{Conclusion:} The scalar AR(p) process is Markov of order \(p\), but its state-space representation defines a first-order Markov process. This Markov structure is essential for the application of the Kalman filter.


% 21. August 2025 ----- Notizen von \cite{palomar:portfolio_optimization}

\subsection{Financial Data: Time Series Modeling}
\subsubsection{Mean Modeling}
- Given past observations we are interested in forecasting the future values of a financial time series
- decide whether to focus on time series of the prices or the returns
- returns tend to be more constant and easier to model 
- prices tend to present a trend 
- Recall: log-prices $y_t=\textbf{log }p_t$ and log-return $x_t= y_t-y_{t-1}$

- the objective is to compute the expectation of the future value of the time series at time $t$ based on the past observations $\mathcal{F}_{t-1}$. This is done in order to uncover potential sturcutral patterns, either through $\mathbb{E}[y_t \mid \mathcal{F}_{t-1}$ for log-prices or $\mathbb{E}[x_t \mid \mathcal{F}_{t-1}$ for log-returns. 