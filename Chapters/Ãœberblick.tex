%------------------------------------------------------------------------------
% Meeting 12. Juni 2025
%------------------------------------------------------------------------------

% mit Beispiel zwei dim random walk 
% Problembeschreibung 
% Bayes -> realtime problem

%------------------------------------------------------------------------------
% Hintergrund Wahrscheinlichkeitstheorie, Bayes' rule, conditioning
% einfacheres Beispiel Zeit, Dynamik, Observation, likelihood mit Bayes -> Problem 

% Gold als Testbeispiel: Kann als einfaches Beispiel genutzt werden, später auf komplexere Sets erweitern

% Random Walk Problem:
% Mit Bayes’scher Methode theoretisch berechenbar, aber praktisch problematisch, da Komplexität mit der Zeit t wächst

% Iterative Lösung nötig, um Berechnung für jeden Zeitschritt effizient durchzuführen → Filtering Equations

% Bayesian Filtering & Kalman Filter:
% Wird eingeführt, um bewegte Objekte zu tracken (z. B. Tracking eines Objekts mit Dynamik) ==> gutes Beispiel suchen

% Beispiel: Observation mit Noise – Kalman Filter hilft, den Zustand iterativ zu schätzen

% Struktur der Arbeit:
% Zuerst Bayes’sche Methode erklären (Kapitel 2)
% Dann Kalman Filter als iterative Lösung einführen
% Beispielrechnung zeigen, um die Methode zu veranschaulichen



%------------------------------------------------------------------------------
% Meeting 17. Juni 2025
%------------------------------------------------------------------------------

% Q: replace k with t for the time steps? 
% A: YES

% q könnte v beeinflussen, würde nicht auf einer Linie fliegen
% update: geschwindigkeit und nicht Position -> durch measurement Model ungenau sehen

% Loop schreiben, t in 1:T
% x= transition update
% y= observation
% x und y abspeichern

% Matrixmultiplikation mit numpy

% random sampling, gaussian
% mit numpy machen, falls es nicht funktioniert: scipy für mehr Verteilungen .stats, interface intuitiver

%------------------------------------------------------------------------------

\textcolor{orange}{noch umformulieren}


%------------------------------------------------------------------------------
% Meeting 22. Juli 2025
%------------------------------------------------------------------------------

% energy_function muss ich nicht coden, wird schon in filterpy gemacht
% Kalman filter package benutzen

% energy_function hänge von theta ab, und theta hängt von den Daten ab
% allgemeine Rekursion machen mit filtering, Filter mit Parameter die mich interessieren initialisieren, score = 0 setzen und durch Daten iterieren und log_likelihood benutzen

% es gibt keine Methode, die die energy_function raus gibt, muss trotzdem Filter laufen lassen und extrahieren und zusammen zählen

% Funktion mit einem Argument -> numpy array sein, Teil von diese, array extrahieren um in der Funktion zu brauchen 


%------------------------------------------------------------------------------
% Meeting 19. August 2025
%------------------------------------------------------------------------------

